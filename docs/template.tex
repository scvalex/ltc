\documentclass[11pt, a4paper, titlepage]{article}
%% \usepackage{fullpage}

\usepackage[T1]{fontenc}
\usepackage{lmodern}
\usepackage{amssymb,amsmath}

\usepackage[hyphens]{url}
\urlstyle{same}
\usepackage[
  setpagesize=false, % page size defined by xetex
  unicode=false, % unicode breaks when used with xetex
  xetex
]{hyperref}

\hypersetup{
  breaklinks=true,
  bookmarks=true,
  colorlinks=false,
  pdfborder={0 0 0}
}

\usepackage{natbib}
\bibliographystyle{abbrvnat}

% Provides \textsubscript
\usepackage{fixltx2e}

\usepackage[osf]{libertine}

% Use microtype if available
\IfFileExists{microtype.sty}{\usepackage{microtype}}{}

\usepackage{graphicx}

% We will generate all images so they have a width \maxwidth. This means
% that they will get their normal width if they fit onto the page, but
% are scaled down if they would overflow the margins.
\makeatletter
\def\maxwidth{
  \ifdim\Gin@nat@width>\linewidth\linewidth
  \else\Gin@nat@width\fi
}
\makeatother

\usepackage{courier}
\usepackage{listings}
\usepackage{float}
\lstset{basicstyle=\small\ttfamily,
        showstringspaces=false}

\let\Oldincludegraphics\includegraphics
\renewcommand{\includegraphics}[1]{\Oldincludegraphics[width=\maxwidth]{#1}}

\usepackage{color}
\usepackage{fancyvrb}
\DefineShortVerb[commandchars=\\\{\}]{\|}
\DefineVerbatimEnvironment{Highlighting}{Verbatim}{commandchars=\\\{\},fontsize=\small}
\newenvironment{Shaded}{\begin{list}{}{\setlength{\leftmargin}{0.5cm}}\item}{\end{list}}
\newcommand{\KeywordTok}[1]{\textcolor[rgb]{0.00,0.44,0.13}{\textbf{{#1}}}}
\newcommand{\DataTypeTok}[1]{\textcolor[rgb]{0.56,0.13,0.00}{{#1}}}
\newcommand{\DecValTok}[1]{\textcolor[rgb]{0.25,0.63,0.44}{{#1}}}
\newcommand{\BaseNTok}[1]{\textcolor[rgb]{0.25,0.63,0.44}{{#1}}}
\newcommand{\FloatTok}[1]{\textcolor[rgb]{0.25,0.63,0.44}{{#1}}}
\newcommand{\CharTok}[1]{\textcolor[rgb]{0.25,0.44,0.63}{{#1}}}
\newcommand{\StringTok}[1]{\textcolor[rgb]{0.25,0.44,0.63}{{#1}}}
\newcommand{\CommentTok}[1]{\textcolor[rgb]{0.38,0.63,0.69}{{#1}}}
\newcommand{\OtherTok}[1]{\textcolor[rgb]{0.00,0.44,0.13}{{#1}}}
\newcommand{\AlertTok}[1]{\textcolor[rgb]{1.00,0.00,0.00}{\textbf{{#1}}}}
\newcommand{\FunctionTok}[1]{\textcolor[rgb]{0.02,0.16,0.49}{{#1}}}
\newcommand{\RegionMarkerTok}[1]{{#1}}
\newcommand{\ErrorTok}[1]{\textcolor[rgb]{1.00,0.00,0.00}{\textbf{{#1}}}}
\newcommand{\NormalTok}[1]{{#1}}
\newcommand{\FIXME}[1]{\textcolor[rgb]{1.00,0.00,0.00}{\textbf{FIXME}}\quad #1}

\makeatletter
\g@addto@macro\@verbatim\small
\makeatother

% Make links footnotes instead of hotlinks:
\renewcommand{\href}[2]{#2\footnote{\url{#1}}}

\usepackage[normalem]{ulem}
% avoid problems with \sout in headers with hyperref:
\pdfstringdefDisableCommands{\renewcommand{\sout}{}}

\setlength{\parindent}{0pt}
\setlength{\parskip}{6pt plus 2pt minus 1pt}
\setlength{\emergencystretch}{3em}  % prevent overfull lines

\newcommand{\mytitle}{LTc: A replicated data store for high-latency disconnected environments\\[0.4cm]Project Report}
\newcommand{\myauthor}{Alexandru Scvor\c tov}
\newcommand{\myemail}{\email{<as10109@doc.ic.ac.uk>}}

\newcommand{\email}[1]{\nolinkurl{#1}}

\begin{document}

\begin{titlepage}

  \vspace*{\fill}

  \begin{center}
    \textsc{Imperial College of Science, Technology and Medicine}\\[0.1cm]
    \textsc{Department of Computing}

    \vspace{3cm}

    \textsc{\Huge LTc}\\[0.2cm]
    \textsc{\Large A replicated data store for high-latency disconnected environments}\\[1cm]
    {\Large Project Report}\\
    \textsc{June 2013}

    \vspace{7cm}

    \textsc{\Large \myauthor}\\[0.1cm]
    \myemail

    \vspace{1cm}

    \textit{supervised by}\\
    \textsc{Susan Eisenbach} \email{<susan@doc.ic.ac.uk>}\\
    \textsc{Tristan Allwood} \email{<tora@doc.ic.ac.uk>}
  \end{center}

  \vspace*{\fill}

\end{titlepage}

\newpage\null\thispagestyle{empty}\newpage

$body$

\clearpage
\bibliography{references}
\addcontentsline{toc}{section}{References}

\clearpage
\appendix

\section{User Guide}
\label{sec:user-guide}

The following is a reformatted version of the \texttt{README.md} from
the source code repository.
\href{https://github.com/scvalex/ltc/blob/master/README.md}{\emph{That}}
file is almost certainly more up-to-date than the following text at
any time.

\subsection{Build}

LTc requires \texttt{ghc} and \texttt{cabal-install} to start
building, and downloads the rest of the dependencies as it needs them.

The simplest way to build LTc is through the \texttt{Makefile}:

\begin{verbatim}
% make
\end{verbatim}

Alternatively, you can explicitly call \texttt{cabal-install}:

\begin{verbatim}
% cabal install --enable-tests
\end{verbatim}

A few tests are always run after every build by the \texttt{Makefile}.
In order to run the full test suite, use one of the following commands:

\begin{verbatim}
% make test
% cabal test
\end{verbatim}

\subsection{Run}

The simplest way to run LTc is through the debugging program
\texttt{ltc}. For instance, to start a node with the Redis adapter
enabled, run:

\begin{verbatim}
% ./ltc redis
\end{verbatim}

You can now connect to the node with any compatible Redis client.

Alternatively, you can start a test node with the web interface enabled
by running:

\begin{verbatim}
% ./ltc node
\end{verbatim}

You can now see the web interface at \url{http://localhost:5000}.

See the help options for the \texttt{ltc} program for details.

\subsection{Embed}

LTc is normally meant to be used as an embedded data store, but an
example use is beyond the scope of this readme. See \texttt{ltc-tool.hs}
and the programs in \texttt{examples/} for code samples.

\end{document}
